\documentclass[11pt]{article}

\usepackage{fullpage}
\usepackage{parskip}
\usepackage[utf8]{inputenc}
\usepackage{xltabular}
\usepackage{fouriernc}
\usepackage{booktabs}
\usepackage{Rcolors}
\usepackage{RlogoNew}

\usepackage[yyyymmdd]{datetime}
\renewcommand{\dateseparator}{-}


% \title{Some \RStudio  Keyboard Shortcuts and Tricks}
% % \author{Bill Venables}
% % \date{\today}
% \author{}
% \date{}
\begin{document}
%\maketitle
\begin{center}
  \Large Some \RStudio  Keyboard Shortcuts, Tips and Tricks
\end{center}
\begin{description}
\item[NOTE:] The prefixes given here are for \Windows or \Linux
  operating systems and keyboards.\\
  The equivalent prefixes for \MacOS are:

  Use \verb|Option | for \verb|Alt|,\\
  Use \verb|Command| for \verb|Ctrl|\\
  Use \verb|Shift  | for \verb|Shift|.
\end{description}


\begin{xltabular}{1.0\linewidth}[!htp]{@{}lXX@{}}
  \toprule
  \textbf{Key} & \textbf{Source Pane} & \textbf{Console Pane}\\
  \midrule Alt+Shift+K & Show keyboard shortcut table &
  Works outside the pane system.\\
  Ctrl+1 & Shift focus to the source pane&
  Works outside  the pane system\\
  Ctrl+2 & Shift focus to the console pane& Works outside
  the pane system\\
  Ctrl+Shift+K & `Knit' the current document to either an html, Word
  or pdf.  (If the current document is an \R script, a temporary
  \texttt{.Rmd} document is constructed from it on the fly.)  & Works
  outside
  of the pane system\\
  Ctrl+Shift+N & Start a new, unnamed \R script in the source pane&
  Works outside the   pane system\\
  (Mouse clicks) & A single click locates the cursor.\par
  When the cursor is within a word, a double click will select the
  entire word.  A third click will select the entire line. & Works
  similarly to source pane\\
  & When the cursor is located \emph{immediately after} an opening
  group delimiter: \texttt{(, [, or \{}, a double click will select
  all text up to, but not including the \emph{matching} closing
  delimiter. & (This handy feature may be only available in the
  preview version of
  \RStudio)\\
  (Group delimiters) & To insert matching group delimiters in code or
  documents:\par
  First select the entire section to be enclosed within the group.\par
  Then press the required delimiter: \texttt{(, [, or \{}\par
  Delimiters are inserted before and after the selected text and the
  cursor located \emph{after} the enclosed group.& Works similarly to
  source pane\par
  Note that this also works for inserting matching quote delimiters,
  \texttt{"}\dots\texttt{"} or \texttt{'}\dots\texttt{'}\\
  Arrow keys: $\uparrow$, $\downarrow$ & Move around the document in
  an obvious way. & Move up and down the command history trail for
  command recall.\par
  Note particularly that if you start typing a command, Ctrl+$\uparrow$
  will limit the history recall to all previous commands
  that \emph{began} in precicsely the same way.\\
  \midrule Esc & & Interrupt the currently executing \R command,
  (if possible!)\\
  Ctrl-Shift-F10 & Restart the \R process, reloading the current
  workspace, but clearing the search path back to the
  default.&  Works outside the pane system\\
  Ctrl+Q & Quit \RStudio altogether &
  Works outside the pane system\\
  \midrule
  Ctrl+A & Select entire document & Select current line\\
  Ctrl+C & Copy selected text & Copy selected text \\
  Ctrl+Shift+C & Comment or uncomment selected lines (a
  toggle) & \\
  Ctrl+D & Delete all on the current line & Delete the
  line being typed \\
  Ctrl+F & Find and replace & \\
  Ctrl+Shift+F & Find all occurrences of a string in the documents
  held in the current project folder, or sub-folders & Works outside
  the pane system.\par (Little known, but very useful!)\\
  Ctrl+I & Indent the current line and move to the next
  line.  When held down, this works its way down the document & \\
  Ctrl+K & Kill to end of line & Kill to end of line \\
  Ctrl+L & & Clears the console pane\\
  Ctrl+Shift+M & Insert pipe, \texttt{ \%>\% }& Insert pipe, \texttt{
    \%>\% }\\
  Ctrl+S & Save the current document & \\
  Ctrl+Shift+S & Save the current document and \texttt{source()} it in
  the console pane& \\
  Ctrl+X & Cut and copy selection & Cut and copy selection, but
  limited to the  line currently being typed\\
  Ctrl+Z & Reverse the previous change.  When typed repeatedly, will
  reverse several previous changes. & Same action, but limited to the
  line currently being typed\\
  Alt+- & Insert assignment operator, \texttt{ <- } & Insert
  assignment operator, \texttt{ <- }\\
  Ctrl+Enter & Execute current line, group or selection in the console
  panel & \\
  Alt+Shift+I & In an \texttt{Rmd} document, insert a code chunk at
  the present  point. & \\
  Ctrl+Alt+Shift+R & In an \R script, insert an \texttt{roxygen2} comment
  skeleton.\par
  The cursor must be at the first line of a function definition, and
  the definition itself must start with a line of the form\par
  \texttt{name <- function(...) \{...}\par
  The opening \texttt{\{} is required.&\\
  \midrule Alt+<letter> & Open menu or sub-menu indicated by
  \underline{<letter>} on the menu bars& Works outside
  the pane system\\
  \bottomrule
\end{xltabular}

\begin{footnotesize}
  NOTE: Keyboard shortcuts may be modified from the menu:
  \begin{center}
    Tools $\longrightarrow$ Modify keyboard shortcuts\dots (or
    equavalently \verb|Alt-T|, \verb|Alt-M|)
  \end{center}
  You may want to change the \texttt{roxygen} generator from
  \verb|Ctrl+Alt+Shift+R| to \verb|F12|, (which is currently unused).
\end{footnotesize}

\end{document}

%%% Local Variables:
%%% mode: latex
%%% TeX-master: t
%%% End:
