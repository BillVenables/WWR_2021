\documentclass[12pt]{article}

\usepackage{fullpage}
\usepackage{parskip}
\usepackage[utf8]{inputenc}
\usepackage{fouriernc}
\usepackage{tabularx}
\usepackage{RlogoNew}
\usepackage{Rcolors}
\usepackage{hyperref}
\hypersetup{
    colorlinks=true,
    linkcolor=cyan,
    filecolor=magenta,      
    urlcolor=blue,
}

\title{Working with \Rlogo:\\
  A University of Queensland Advanced Workshop\\[10pt]
Preliminaries for Participants}

\author{Bill Venables, Data61, CSIRO\\
Rhetta Chappell, Griffiths University}
\date{2021-02-02 -- 2021-02-05\\[10pt]
  Room 216, Sir Llew Edwards Building\\[5pt]
  The University of Queensland}


\begin{document}
\maketitle

\section{Introduction and welcome}
\label{sec:welcome}

This four-day workshop is intended for current \R users who wish to
deepen their understanding of the programming environment and of the
data analysis and visualisation tools it provides.  Some experience
with \R is assumed, but some experience with data analysis and
programming in professional practice is probably just as important.

The workshop will involve formal presentations, interspersed with
periods of free time for further investigation of the examples
presented, and any related questions that arise.  Participants are
\emph{encouraged to bring along their own data sets} and data analysis
problems, and to discuss them with the presenters during the free
periods.  The best way to learn \R is to \emph{use} \R on real
problems from your own professional disciplines.


\section{Materials and preparation}
\label{sec:prep}

\begin{description}
\item[Laptop] To take part in the sessions, participants will need to
  bring along their own laptop with \R installed beforehand.
  
\item[\R version] The version of \R installed should be the latest,
  \emph{\R~4.0.3}.

  Versions of \R older than \R~3.5.0 will not be able to
  handle some of the course data sets.

  [Note that going from \R~3.*.* (or older!) to \R~4.0.* is a major
  upgrade requiring all your currently installed packages to be
  re-installed. This may take some time.]

\item[CRAN Packages] Participants should have the required
  capabilities to \emph{install and maintain} \R packages on the
  laptop they will be using for the workshop.

  Participants are encouraged to \emph{update their packages} to the
  latest versions available.

  The workshop will use a number of additional \R packages, and
  an \R script, namely \verb|WWR_packages_2021.R|, will be provided,
  \emph{by email}, to install these beforehand. The simplest way to
  to use the script is, from within an \R console, issue the command

  \verb|source("WWR_packages_2021.R")|

  making sure the script file itself is in the \R working directory.
  This may be done immediately once the script is available.

  Note that in addition to any required CRAN packages, this script
  will also install several other packages from a \emph{GitHub} repo,
  namely \url{https://github.com/BillVenables}.  The software to do
  this, if needed, will be installed automatically.

  
\item[RStudio] The \RStudio API (`application-programmer interface')
  to \R will be assumed and used extensively throughout the
  workshop.\footnote{Other interfaces to \R could be used if
    participants have a particular preference, but some aspects of the
    work will then not be easily available.  We strongly recommend
    participants have \RStudio installed, whether they use it
    routinely or not.}  As of now, the latest version is
  \RStudio~1.4.1103.  We recommend you update to this release, (or
  later).  A link to the download site is given
  \href{https://rstudio.com/products/rstudio/download/}{here}. The
  free desktop version is adequate.

  To update an existing installation, a good way to begin is to start
  up \RStudio and go to the menu:
  \begin{center}
    \texttt{Help} $\longrightarrow$ \texttt{Check for updates}
  \end{center}
  and follow the advice.
  
  The main materials for the course will be delivered as a complete
  \RStudio~\emph{Project} as a \verb|.zip| file.  This can be
  downloaded and expanded to provide a directory (or folder) that will
  form the \emph{working directory} for the workshop.  This directory
  will have several sub-directories with the file materials logically
  classified.  When expanded, the folder name will be \verb|WWR_2021|.

  Users may locate this folder anywhere on the file system of their
  machines, or even on an attached memory
  device.\footnote{Participants unfamiliar with, or preferring not to
    use \RStudio, may use this project directory in the same way they
    would any \R working directory.}
\item[Other tools used with \R] To take part fully in the workshop,
  some ancillary programming tools will be needed.  These are mainly
  needed to deal with using fast compiled code in computations and
  package construction. These come in several forms, depending on the
  operating system on your machine:
  \begin{description}
  \item[\Windows] With \R~4.0.0 and later the compiler tools are
    available as a \Windows~\verb|.exe| file, which can be downloaded
    from \href{https://cloud.r-project.org/bin/windows/Rtools/}{this
      link}.  In addition to running the \verb|.exe| file, some
    additional steps may be required, as outlined on the download
    page.
  \item[\MacOS] On this system \R is built using
    \verb|Apache Xcode 10.1| and \verb|GNU Fortran 8.2|, which are
    required for compiled code within \R itself.  See
    \href{https://cloud.r-project.org/bin/macosx/tools/}{this link}
    for further details. (Disclaimer: Neither presenter has any
    experience with \MacOS!)
  \end{description}
\item[Version control] Though not a fundamental part of the workshop,
  participants will be encouraged to learn, and to use, version
  control.  For this purpose the currently favoured tool is
  \git\footnote{The name is allegedly derived from the word
    `\emph{git}', which means `\emph{stupid person}' in British slang.
    An alternative derivation is an an acronym for `\emph{Global
      Information Tracking}'.  Take you pick.}  This is free and
  open-source software, available for all common operating systems,
  and freely located on the web.  Installation is simple.

  Eventually you may wish to get yourself a free \github (or \gitlab)
  account to facilitate collaborative working, or to allow access on
  several machines.  Details are left for the reader to investigate.
  
\item[\TeX{} and \LaTeX] The \TeX{} document preparation system, which
  includes \LaTeX{} and many other tools, is a major system and
  although it is free and available for all operating systems,
  installation can be a major undertaking.  It also comes in several
  different distributions, which can also be a complication.

  The system is mainly needed in \R for rendering documents into
  \verb|pdf| (rather than \verb|html| or \verb|Word|, for example) and
  for building some optional parts of \R packages.  Users for which
  this is important will clearly need to have it installed, but for
  this workshop it is optional.  To some extent the \R package
  \verb|tinytex| provides enough functionality for basic applications,
  if users unfamiliar with the system wish to investigate.
  

\end{description}

\section{Synopsis}
\label{sec:synopsis}

The recommended set-up process is:
  \begin{enumerate}
  \item Make sure you have the correct write capabilities on your
    laptop to update \R and install and update packages.

    Make sure you are connected to the internet.
  \item Update your \R to \R~4.0.3, if necessary, and update your
    currently installed packages to the latest CRAN versions.
    
  \item Make sure you have \RStudio installed and upgraded to the
    latest release.
    
  \item From within an \R session, run the script
    \verb|WWR_packages_2021.R| to install any required new packages
    you will need for the workshop.
    
  \item Download the file \verb|WWR_2021.zip| from the link below and
    expand it to provide the \verb|WWR_2021| folder, where you wish to
    locate it for use as the working directory during the workshop.
    \begin{description}
    \item[NOTE 1:] This {\tt .zip} folder will be
      available on memory sticks at the workshop itself.  So if there
      is any problem, apart from updating \R itself, it can be fixed
      at the workshop itself as necessary.
    \item[NOTE 2:] Users with \git and some experience with using it
      may prefer to clone the \github repo directly.  The
      relevant address is
      \begin{center}
        \url{https://github.com/BillVenables/WWR_2021}
      \end{center}
      Alternatively the folder can be downloaded as a \verb|.zip| from
      this same address.  You should be able to go to the tab:
      \begin{center}
        \texttt{Code} $\longrightarrow$ \texttt{Download ZIP}
      \end{center}

    \end{description}
  \item This should complete your preparations for the workshop.  If
    you need further help you are free to contact Bill Venables at
    this email address:
    \href{mailto:Bill.Venables@gmail.com}{Bill.Venables@gmail.com}.
 
  \end{enumerate}

        

\section{Links}
\label{sec:links}

<section yet to come>

Good luck, and welcome to the workshop!




\end{document}

https://www.dropbox.com/s/voqqr53v1amnjmk/WWR_2020.zip?dl=0

https://www.dropbox.com/s/u4cy0ftfkucchjc/WWRData_0.1.0.tar.gz?dl=0
https://www.dropbox.com/s/8dsho00z4c0ii1k/WWRGraphics_0.1.2.tar.gz?dl=0
https://www.dropbox.com/s/0vk38t5lkr6wfa4/WWRUtilities_0.1.2.tar.gz?dl=0
https://www.dropbox.com/s/dvn4fn7231vmyf6/WWRCourse_0.2.2.tar.gz?dl=0

https://www.dropbox.com/s/2ovxsjx9cry2net/WWR_install_packages.R?dl=0

%%% Local Variables:
%%% mode: latex
%%% TeX-master: t
%%% End:
